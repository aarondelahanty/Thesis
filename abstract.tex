\begin{abstract}


%Background:
Microfluidic devices generally require precise control of fluid flowrates in order to accurately and reliably perform their various functions. In the case of droplet-makers fluid flow may be manipulated to dictate the size, frequency and distribution of droplets. Multiple approaches may be taken in order to control fluid flow in such devices. Here a pressure-driven flow controller (PDFC) is developed and characterized for use as a flow provider for droplet-makers and as a tool for further microfluidics-based research.

%Objectives:
	Previously, droplet-makers that utilize volumetric flow control have been used to define the relationship between continuous and discontinuous phase flowrates and the resulting droplet parameters and flow regimes. Here pressure-driven droplet-makers are characterized and compared to the existing systems.
Furthermore, an investigation into the ability to control droplet formation in real-time is conducted by quantifying the time-to-stability of the electronic flow controller and droplet-maker system.

%Methods:
The work conducted and presented here can be divided clearly into three sections (i) design, fabrication, development of the PDFC (ii) characterization of the PDFC system and (iii) investigation into the behavior of droplet-maker devices as driven by pneumatic pressure.

%Results:
	The PDFC system consists of a micro controller, sensing pressure transducers, regulating pressure transducers, as well as various pneumatic and electronic components used to integrate the system into University of Glasgow's Franke Lab's existing microfluidic test set-up. The system was characterized to show pressure control ranging from 0 to 1000mbar with 4 discreetly controlled channels capable of  precision of XX with a time response of XX. When applied as the flow controller for droplet-makers it was found that droplet formation as defined by length:width behaved similarly to previously volumetric flow rate systems.

%Limitations:
A major limitation of pressure-driven flow systems is that the flow rate within microfluidic devices varies as a function of device geometry. Herein significant discussion is presented as to methods of theoretical and experimental approximations of the volumetric flowrates resulting from pressure-driven flow.


\end{abstract}