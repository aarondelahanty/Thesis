\begin{abstract}


%Background:
Microfluidic devices generally require precise control of fluid flowrates in order to accurately and reliably perform their various functions. In the case of droplet-makers fluid flow may be manipulated to dictate the size, frequency and distribution of droplets. Multiple approaches may be taken in order to control fluid flow in such devices. Here a pressure-driven flow controller (PDFC) is developed and characterized for use as a flow provider for droplet-makers and as a tool for further microfluidics-based research.

%Objectives:
Previously, droplet-makers that utilize volumetric flow control have been used to define the relationship between continuous and discontinuous phase flowrates and the resulting droplet parameters and flow regimes. Here pressure-driven droplet-makers are characterized and compared to the existing systems.

%Results:
The PDFC system was characterized to show pressure control ranging from 0 to 1000mbar with 4 discreetly controlled channels capable of $\pm 1$ mbar accuracy with a signal standard deviation of $0.25$ mbar. The PDFC was then used to to drive flow in a T-junction device geometry. Here, for the first time the transition from dripping to squeezing droplet formation regimes is clearly documented in a pressure-driven flow system. The resulting droplets show high monodispersity, less than 1\% variation as expressed by droplet length standard deviation, $\sigma$, over mean droplet length, $\mu$.


\end{abstract}