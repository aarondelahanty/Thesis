\chapter{Arduino Code}
\label{chp:arduino}
\scriptsize

\begin{lstlisting}

    
//Aaron Delahanty
//2224135D@student.gla.ac.uk
//University of Glasgow
//21Oct2015
//Electronic Flow Controller

//This application (sketch) is intended to be the sole sketch responsible for measuring 
//and regulating pressure in the Electronic Flow Control (EFC) system being developed 
//for T.Franke's microfluidic group. The objective of the sketch is to faciliate
//communication with two types of integrated chips (i) Analog Microselectronic's AMS5915
//pressure sensors,and (ii) Microchip Technology Inc.'s MCP4921 DAC.

//In addition to this sketch the system will use a LabView VI to set the desired 
//pressure values by DAC, and display the measured pressure for the 4 discrete channels
//This communication will be facilated by serial communication over USB.

//Revision Notes:
//02192016  - Removed Volmeter functionality
//          - Removed All Delays
//          - 4th Channel DAC is faulting (output is always 1V)


//Wire.h is the arduino library for I2C/TWI comm
#include <Wire.h>
#include <SPI.h>

//Define variables used for conversion of pressure units from raw counts mbar.
    int p_min;
    int p_max;
    int digoutp_min;
    int digoutp_max;

//Define array for pressure channel comaprison and final output
//Comparison is used to determine whether the high range 1000mbar or high resolution 
//350mbar sensor should be used
    float p_comp[8];
    float p_out[4];

//Prior to sending the 12bit input which will drive the T2000 pressure regulators
//the DAC needs to receive 4 configuration bits
//Bit 15 - 0 = write to DAC, 1 = ignore command
//Bit 14 - BUF; 1 = buffer v_ref, 0 = bypass buffer
//Bit 13 - GA; 1 = v_out is 1X v_ref, 0 = vout is 2X v_ref
//Bit 12 - SHDN; 1 = active, 0 = shutdown DAC

//Default configuration bits set to [0,1,1,1]
    unsigned int Config = B0111;

void setup()
{
    //Set baud rate to 9600 
    Serial.begin(9600);
    // join i2c bus
    Wire.begin(); 
    // join SPI bus
    SPI.begin();
}

void loop()
{

//START DAC CODE:

 // check serial 

  if ( Serial.available() ){
    // cast the string read in an integer 
    String p_write = Serial.readString();

  int commaIndex1 = p_write.indexOf(',');
  int commaIndex2 = p_write.indexOf(',',commaIndex1+1);
  int commaIndex3 = p_write.indexOf(',',commaIndex2+1);


  String P1s = p_write.substring(0,commaIndex1);
  String P2s = p_write.substring(commaIndex1+1,commaIndex2);
  String P3s = p_write.substring(commaIndex2+1,commaIndex3);
  String P4s = p_write.substring(commaIndex3+1);

  unsigned int a[3];
  a[0]=P1s.toInt();
  a[1]=P2s.toInt();
  a[2]=P3s.toInt();
  a[3]=P4s.toInt();


    float cs[3];
    for (int i = 0; i <= 3; i++)
       {
        cs[i] = i+7;
        pinMode(cs[i], OUTPUT);
        

        
        //Command is the variable sent to DAC to drive pressure regulator
        //The commmand variable is comprised of the 4 configuration bits and 12 data bits
        unsigned int command;
        //Shift the config bits 12 positions leftward, augment with the a[i] pressure value 
        //in volts
        command = ( Config << 12 | a[i]);

        //Write command to each ith DAQ
        SPI.beginTransaction(SPISettings(20000000, MSBFIRST, SPI_MODE0));
        //Set ith output pin to LOW, in order to accept write command
        digitalWrite(cs[i], LOW);
        //SPI.transfer function is only capable of sending 8bits at a time
        //The 16bit command is therefore split into a high and low byte
        int high = highByte(command);
        int low = lowByte(command);
        SPI.transfer(high);
        SPI.transfer(low);
        SPI.endTransaction();
        //Set ith pin HIGH to end write command
        digitalWrite(cs[i], HIGH);
        //Optional print for debugging
        //Serial.println("16bit command sent:");
        //Serial.println(value,BIN);
        }
  }


//END DAC CODE

//START PRESSURE SENSOR CODE:

  //Define buffer to hold sampled values
  byte buffer[4];


  for (int addr = 1; addr <= 8; addr++)
    {
        //Pressure sensor addresses have been configured to 1 through 8.
        //Sensors 1,2,3,4 are high range 1000mbar
        if (addr <= 4)
          {
            //Set conversion constants for AMS 5915-1000-D
            p_min = 0;
            p_max = 1000;
            digoutp_min = 1638;
            digoutp_max = 14745;
          }

        //Sensors 5,6,7,8 are high resolution 350mbar
        else
          {
            //Set conversion constants for AMS 5915-350-D-B
            p_min = 0;
            p_max = 350;
            digoutp_min = 1638;
            digoutp_max = 14745;
          }

        //Wire.requestFrom(address, quantity)
        //address: the 7-bit address of the device to request bytes from
        //quantity: the number of bytes to request
        //request four bytes from sensor (first 2 are pressure, second 2 are temp)
        int n = Wire.requestFrom (addr, 4);
    
        //Ensure response is 4byte
        if(n == 4)
          {
          
            Wire.readBytes (buffer, 4);
        
            unsigned int p_raw = word (buffer[0], buffer[1]);     // word(high,low)
            unsigned int t_raw = word (buffer[2], buffer[3]);
        
            // note that the next bit operations works best with "unsigned int", not with "int"
            p_raw &= 0x3FFF;    // 14bits pressure data
            t_raw >>= 5 ;       // 11bits temeperature data, shift them in position
        
            
            // convert raw pressure to mbar (equation and constants from datasheet)
            float pressure = ((( (float) p_raw - digoutp_min ) / ((digoutp_max - digoutp_min) / (p_max - p_min ))) + p_min );
        
            // convert raw temperature to degC (equation and constants from datasheet)
            float temperature = (( (float) t_raw * 200.0 ) / 2048.0 ) - 50.0;
        
            // with the pressure comparison array with all sensor's outputs
            p_comp[addr-1] = pressure;
          }

        //If no response is present, or response is not 4bytes
        else
          {
          Serial.println ("Error, no sensor found");
          }
    }

      
   int i;
   for (i = 0; i <= 3; i++)
     {
        if (p_comp[i]> 350) 
          {
            p_out[i] = p_comp[i];
          }
        else 
          {
           p_out[i] = p_comp[i+4];
          }
        Serial.println(p_out[i]);
     }
// END PRESSURE SENSOR CODE

}
    \end{lstlisting}

