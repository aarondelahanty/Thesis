\chapter{Introduction}

The study of microfluidics requires by definition generation of fluid flow. In research and experimental settings syringe pumps are commonly used as flow providers\cite{Christopher2008}. They offer the advantage of being capable of directly controlling the rate of flow in units of volume per time. Furthermore, programmable syringe pumps have been developed that allow for modulation of flowrate as a function of time. An alternative to syringe pump systems is pressure-driven flow in which precision regulated compressed gas is used to pressurize fluid-filled reagent reservoirs, which in turn drives flow through small diameter tubing and subsequently any applicable microfluidic device. A custom developed pressure-driven flow system may offer advantages such as increased flow stability, capability to support high numbers of parallel flow channels, and flowrate modulation response time \cite{Bong2011, Lim2015}. This project focus on the development of such a system, herein referred to as a Pressure Driven Flow Controller (PDFC). 

The work documented here can be divided into two primary categories. First, the PDFC system is designed, built, and characterized. Secondly, the developed device is put to use to characterize micro-droplet formation in T-junction geometry operating under pressure-driven flow. The thesis presented herein is to be divided accordingly. First the design, development, and characterization of the flow device is presented. Followed by a presentation of the background theory necessary to discuss droplet formation. Experimental observations are presented and discussed. Finally, a conclusion is presented that describes the major accomplishments and courses of further investigation.

The custom developed microcontroller code, LabVIEW Virtual Interface (VI), and an operational user manual are included in the appendices, should further research be conducted using the system.